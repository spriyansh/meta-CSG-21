\begin{abstract}

\textcolor{red}{The slow rate of evolution 1500 bp long 16s-rRNA gene makes it a perfect nominee for taxonomic surveys. Metagenomic studies allow direct environmental sampling of the microbial communities, which provides a comprehensive view of both the microbial genome and biological interactions. The graphical form of these pairwise associations is called a microbial co-occurrence network. The nodes of the networks denote the microbial species, and the edges of the network describe the statistically significant association. These co-occurrence networks also help determine the critical microbial species or hubs dominating a particular community. The addition of phylogenetic measures to microbial co-occurrence networks increase interpretability and allows generating ecologically sound conclusions. Using the exact sequence, one can detect microbes that may have diverged a million years ago, whereas close-reference clusters have greater interpretability across the studies. However, whether the use of exact sequence instead of clustered sequences would add more variability to the co-occurrence network remains unanswered. Thus current review addresses the fundamentals of metagenomic studies. In addition, a detailed study of microbial co-occurrence is discussed, accompanied by molecular phylogenetics. The review aims to reflect on the widely used strategies and concepts in microbial ecology.}

\end{abstract}
