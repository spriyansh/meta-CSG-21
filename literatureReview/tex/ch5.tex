\chapter{Phylogenetic Measures}
\pagenumbering{arabic} \setcounter{page}{16}

This section discusses the four most widely used approaches that are used in designing the phylogenetic trees. The approaches are classified distance based-approaches and character-based approaches.

\subsection{Distance-based approaches}

As the name suggests, the distance-based algorithms utilise the matrices containing pairwise distances. Pairwise distances aid in the construction of trees via Bayesian and likelihood methods. The two most widely used methods which utilise the pairwise distance are the Unweighted Pair Group Method with Arithmetic Mean (UPGMA) and Neighbour Joining (NJ) method \cite{munjal_2019_phylogenetics}. UPGMA is an agglomerative hierarchical clustering method that builds a rooted phylogenetic tree by assuming equal rates of evolution. The NJ is an iterative clustering method that produces an unrooted tree by considering different rates of evolution. Given that both the methods use a distance matrix, they lower computing time for large datasets compared to character-based methods. NJ is preferred over UPGMA for almost all cases as the previous assumes the same evolution rate for all of its lineages \cite{munjal_2019_phylogenetics}.

\subsection{Character-based approaches}

The character-based approaches use actual sequence alignments similarities to calculate the distances. The character-based methods are more greedy for computational resources and time as compared to the distance-based techniques \cite{munjal_2019_phylogenetics}. However, they generate exact phylogenetic trees. The character-based methods include Maximum-Likelihood (ML) and Maximum-Parsimony (MP). The MP is based on the assumption that the simplicity extends that the most parsimonious tree would be the one that reflects the slightest evolutionary changes \cite{munjal_2019_phylogenetics}. The ML, on the other hand, utilises the probabilistic modelling based on Markov chains to derive the trees.

\subsection{Bootstrapping}

Once the trees are constructed, the inference of their reliability poses a challenge. This is known as bootstrapping. Bootstrapping in phylogenetic analysis is a greedy approach that picks deviated/pseudo samples from the original dataset to design the tree \cite{lemoine_2018_renewing}. Essentially, running any tree constructing algorithms on a sample data more than 100 times generates a bootstrapping value which is a measure of reliability as it gives the probability of a branch.

\subsection{Microbial Community \& Phylogeny}

Bacterial lineages determine how the microbial communities are formed as the phylogenetic relatedness points towards functional relatedness. Traits of bacteria influencing ecological functions are phylogenetically conserved. High ecosystem functioning is directly related to the co-existence of functionally distinct lineages or from the existence of productive lineages that outperform the rest. Functional differences allow coexistence through niche segregation events and deliver different functions to the ecosystem. A study performed by \emph{Goberna et al} detected that the abundance of divergent lineages in the community increases the ecological processes \cite{goberna_2018_phylogeneticscale}. Therefore to elucidate the ecological functioning of the community, incorporation of phylogenetic measures is very crucial.