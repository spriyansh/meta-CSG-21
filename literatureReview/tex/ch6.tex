\chapter{Co-occurrence Networks}
\pagenumbering{arabic} \setcounter{page}{18}

This section discusses the merits and demerits of microbial co-occurrence networks. It also gives a brief account on widely used measures in co-occurrence networks design, such as Pearson correlation, Spearman Correlation and Bray-Curtis similarity measure.

\section{Co-occurrence Networks}
The microbes do not dwell in isolation; instead, they thrive in colonies and form associations. These associations shape the patterns and structure of their microscopic world, which administers the macroscopic world. For example, the gut microbiome regulates the food choices of an individual. The interaction patterns among these microbes are directed by their evolutionary cycle and inter/intra-species interactions. They can have a positive association like mutualism, commensalism, synergism, or negative associations like competition, parasitism, predation \cite{weiss_2016_correlation}. To analyse these associative patterns, microbiologists have previously implemented the concepts of networks science onto these associations. The graphical form of these pairwise associations is called a microbial co-occurrence network. The nodes of the networks denote the microbial species, and the edges of the network describe the statistically significant association. These co-occurrence networks also help determine the critical microbial species or hubs dominating a particular community.\newline

However, drawing adequate ecological conclusions from an entirely mathematical concept is not advisable. Criticisms have been made about co-occurrence networks for predicting non-trophic interactions, which calls for integrating community-level insights. Studies have shown that processes like habitat filtering should be considered while generating the co-occurrence networks to draw ecologically sound conclusions. \emph{Goberna et al} have compared the effect of habitat filtering, spatial limitation, and biological interactions in governing the community patterns; the study found that habitat filtering and natural interactions are much more predominant than dispersal limitations \cite{ref10}. This might suggest that associations form independent of their geographical location. They also explained the need to consolidate phylogenetic measure into the downstream analysis to make the networks a close imitation of nature. One might overshadow that the microbes can interact with more than one neighbour, which gives triplet or quadrupole interactions, rather than a pairwise interaction \cite{berry_2014_deciphering}.

\subsection{Pearson's Correlation}
The Pearson's Correlation (PC) estimates the magnitude of the linear covariance between two independent variables \cite{schober_2018_correlation}. The data should be randomly sampled and devoid of outliers showing linear patterns in a scatter-visual test. It assumes that data follows normal distribution the values of a variable are not correlated to themselves \cite{schober_2018_correlation}. The test works with continuous data points sampled or for a paired observation. In terms of co-occurrence networks, the microbial pair forms an {x,y} set of statements, given that there is no correlation between $x_{i}$ or $x_{n}$/$y_{i}$ or $y_{n}$, where (i = 1). Pearson's correlation calculates three measures, i.e. Coefficient (r), Coefficient of determination ($R^{2}$), and p-value. The r tells the direction and strength to which the x and y are correlated. The r can range between (-1,1), with -1 suggestive of a strong negative correlation and 1 suggestive of a strong positive correlation. The $R^{2}$ explains the variation shared between the x and y, and it can range between (0,1). Lastly, the p-value measures the evidence against the null hypothesis ($H_{0}$) that there is no correlation between x and y. The working formula boils down to dividing the covariance by the product of the standard deviations,

$$r_{xy} = \frac{\sum_{i=1}^{n} (X_{i} - \bar{X}) * (Y_{i} - \bar{Y})}{\sqrt{\sum_{i=1}^{n} (X_{i} - \bar{X})^{2}} * \sqrt{\sum_{i=1}^{n} (Y_{i} - \bar{Y})^{2}}}$$

\subsection{Spearman's Correlation}
The Spearman's Correlation (SC) estimates magnitude \& direction of the monotonic relation among the two ranked variables \cite{dewinter_2016_comparing}. The SC is implemented on the ordinal data rather than continuous data. It assumes a monotonic association between the variables, i.e. if one is changing, the other remains the same. The SC is well suited for explaining interactions like amensalism. It does not assume the data to be normally distributed and works by ranking the variables first. As the variables are ranked according to their magnitude, they can be implemented on both ordinal and continuous datasets. Pearson's correlation calculates three measures, i.e. Coefficient ($r_{s}$) and p-value. The rs tells the direction and strength to which the x and y are correlated. The $r_{s}$ can range between (-1,1), with -1 suggestive of a perfect negative correlation and 1 suggestive of a perfect positive correlation \cite{dewinter_2016_comparing}. Lastly, the p-value measures the evidence against the null hypothesis ($H_{0}$) that there is no correlation between x and y. The working formula boils down to dividing the Pearson correlation over the (mean) ranks,

$$r_{s} = 1 - \frac{6 * \sum D^{2}}{n^{3} - n}$$

\subsection{Bray Curtis Dissimilarity}
The Bray Curtis Dissimilarity (BC) quantifies the dissimilarity between the species between two different sites. In terms of microbial ecology, one can say it measures the beta-diversity by comparing the alpha-diversity \cite{zhang_2019_the}. It falls between 0 to 1, with 0 suggesting that they are identical, and one is suggestive of 100 per cent dissimilar. The BC dissimilarity assumes that both the sampling sites have either the same size or same volume as the BC does not integrate the notion of space \cite{zhang_2019_the}. The BC can be calculated by dividing the sum of lesser counts of species found in both sites by the sum of the alpha-diversity measure of each site,

$$BC_{ij} = 1 - \frac{2 * C_{ij}}{S_{i} + S_{j}}$$