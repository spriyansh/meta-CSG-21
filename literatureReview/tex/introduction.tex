\chapter{Introduction}
\pagenumbering{arabic} \setcounter{page}{1}

\section{Overview}
Life on earth sprang from microscopic uni-celled organisms roughly around 4 billion years ago. Since then, these tiny life forms have been dwelling in every part of the earth. From extreme abiotic environments to complex life forms, microbial life is omnipresent. Prokaryotic microbes are the principal recyclers of the biosphere and form the largest reserve of nutrients \textcolor{green}{such as} phosphorus (P), Nitrogen (N) and Carbon (C) [1,2]. Essentially they are a vital constituent of nutrient cycles and food chains through which complex life forms are sustained. It has been evaluated that our microbiome size surpasses the total number of our cells [3,4]. Yet, genomic studies on these simple life forms are troublesome to perform because most of these microbes are difficult to culture in the labs and rarely exist in isolation [5]. Owing to these hurdles, even the studies conducted using microbial clonal cultures do not reflect the microbial community's actual biology and communal interactions [2]. However, with the \textcolor{green}{development} in next-generation sequencing (NGS) technology, scientists have succeeded in overcoming quite a few challenges in the field of microbial genomics. This branch of genomics which specifically elucidates the molecular study of those microscopic life forms that are hard to culture, has been assigned the term Metagenomics [6]. Metagenomic studies eliminate the need for clonal cultures and allow direct environmental sampling of the microbial communities (metagenome) [7]. This provides a highly descriptive assay illustrating a comprehensive view of both the microbial genome and biological interactions of the community [7].

\section{Impact of Genomics}
Generally, metagenomic procedures either employ 16s Ribosomal RNA (16s-rRNA) (for eukaryotes, it is 18s rRNA) sequencing methods or Whole Genome Shotgun (WGSS) sequencing methods. Both of these methodologies require trimming, error correction and reference database comparisons [8]. De-novo assemblies which do not require reference database comparisons are also exercised when the reference database is unavailable; however, they need more computational resources. 16s-rRNA genes are of tremendous significance as they hold the highly conserved genes and can be used to generate phylogenetic relations among microbial communities [8]. 16s-rRNA \textcolor{green}{gene} sequencing uses Polymerase Chain Reaction (PCR) to amplify the hypervariable segment (v1-V9) of prokaryotic 16s-rRNA, which generates amplicons that are then multiplexed, i.e. pooled together after applying molecular barcodes. This strategy is also called Amplicon sequencing. In fungal genomes, the Internal Transcribed Spacer (ITR) region is targeted; therefore,  it suffers from a high false-positive rate and host DNA interference [9]. Contrary to that, 16s-rRNA sequencing is affordable and offers a better taxonomic resolution at the genus and species level due to the availability of highly curated datasets. A 2019 study conducted by Gupta et al. also demonstrated that the 16s-rRNA provides more sensitive and comprehensive insights when compared to the traditional culture methods (TCMs). The results reflect that the 16s-rRNA identifies ~75\% unique elements, whereas the TCMs were only able to locate ~23\% bacterial elements [10]. 16s-rRNA sequencing either delivers Amplicon Sequence Variants (ASVs) or Operational Taxonomic Units (OTUs) depending upon the pipeline used for clustering. OTUs are the clusters of sequencing reads generated through a dissimilarity threshold filter. In contrast, the ASVs are exact sequences to a single nucleotide level offering a more acceptable resolution [11]. A more detailed comparison of OTUs and ASVs is discussed later in the review.

\section{Soil Ecology}
\textcolor{red}{Soil ecology is one of the highly reshaped fields of soil microbiology by the addition of NGS technology.} Microbial soil ecology studies how microbes interact; it was previously thought to be a dead area; however, after displacing TCMs with the WGSS/16s-rRNA methods, the area is revolutionised [12]. In the last decades, soil microbiologists have discovered a plethora of novel taxa (phyla, classes, genus) owing to metagenomics [12]. Soil metagenomics attempts to answer some of the requisite ecological questions like how microbial communities form? Or how communities communicate in space-time through signalling? Through international collaborations, soil microbiologists and bioinformaticians are also generating sophisticated datasets to help future scientists perform reference assemblies. In 2010 Earth Microbiome Project was announced which intends to collect and analyse the microbial community of the earth [13]. Similarly, in 2014 Brazillian Microbiome Project and China Soil Microbiome Initiative were announced, which have the related vision of exploring microbial communities [14,15]. Now that we have the tools to look at microbial life \textcolor{green}{to a greater detail}, we can also implement system/network science principles to do a habitat-based examination of the microbial community. By utilising highly conserved 16s rRNA methods and adding molecular phylogeny of the population, we can understand biological interactions adequately [16]. One such approach is the study of co-occurrence networks. Co-occurrence networks can help interpret the effect of interspecies interactions like mutualism and parasitism [17]. Fundamentally, the interactions can be either positive or negative, influencing either aggregation or segregation. The positive interactions include cooperative processes like quorum sensing, whereas the antagonistic interactions include phenomenon like competition [18]. Microbial co-occurrences networks can illustrate the biological interactions well; however, in 2014, \textcolor{green}{b}erry et al. demonstrated that these networks lose interpretability when habitat filtering (i.e. tolerance to local stress) becomes notable. Yet, Goberna et al. conducted a study by employing phylogenetic metrics in co-occurrence networks. They concluded that phylogenetic relatedness could help to explain ecologically essential patterns under the influence of habitat filtering [19]. textcolor{red}{However, the impact of OTUs and ASVs is still an uncharted territory.}\newline \newline The current review addresses the crucial fundamentals of the metagenomic studies, featuring the 16s-rRNA sequencing method. The review further discusses the two widely used 16s-rRNA sequencing pipelines that produce different sequences tables, i.e. \textcolor{red}{ASVs and OTUs.} Lastly, a detailed study of microbial co-occurrence is discussed, accompanied by molecular phylogenetics. The review aims to reflect on the widely used pipelines with phylogenetic metrics and interpret ecological patterns from microbial co-occurrence networks.